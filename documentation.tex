\documentclass[12pt]{article}

\usepackage[a4paper,left = 2cm,right = 5cm,footskip=5mm, bottom = 2cm, top = 2.3cm]{geometry}
\usepackage[utf8]{inputenc}
\usepackage[ngerman]{babel}
\usepackage[T1]{fontenc}
\usepackage{marvosym}
\usepackage{wrapfig}
%\usepackage{MnSymbol}
%\usepackage{mathtools}
%\usepackage{amsfonts}
\usepackage{tikz}
%\usepackage{chngcntr}
\usepackage{graphicx}
\usepackage{bbm}
\usepackage{esvect}
\usepackage{scrlayer-scrpage, lastpage}
\usepackage{chemfig}%Strukturformeln
\usepackage{times}
\usepackage[onehalfspacing]{setspace}
\usepackage[compact]{titlesec}
% Die nächsten vier Felder bitte anpassen:
\newcommand{\Aufgabe}{Aufgabe 3} % Aufgabennummer und Aufgabennamen angeben
\newcommand{\TeamId}{Nemanja Cerovac, Jonathan Rajewicz, Lukas Florian Richter}                       % Team-ID aus dem PMS angeben
\newcommand{\TeamName}{Carl-Friedrich-Gauß-Gymnasium Frankfurt (Oder)}                 % Team-Namen angeben
\newcommand{\Namen}{Jonathan Rajewicz}           % Namen der Bearbeiter/-innen dieser Aufgabe angeben
 
% Kopf- und Fußzeilen
\usepackage{scrlayer-scrpage, lastpage}
\setkomafont{pageheadfoot}{\large\textrm}
\lohead{A3}
\rohead{\TeamId}
\cfoot*{\thepage{}/\pageref{LastPage}}

% Position des Titels
\usepackage{titling}
\setlength{\droptitle}{-2.0cm}

% Für mathematische Befehle und Symbole
\usepackage{amsmath}
\usepackage{amssymb}

% Für Bilder
\usepackage{graphicx}

% Für Algorithmen
\usepackage{algpseudocode}

% Für Quelltext
\usepackage{listings}
\usepackage{color}
\definecolor{mygreen}{rgb}{0,0.6,0}
\definecolor{mygray}{rgb}{0.5,0.5,0.5}
\definecolor{mymauve}{rgb}{0.58,0,0.82}
\lstset{
  keywordstyle=\color{blue},commentstyle=\color{mygreen},
  stringstyle=\color{mymauve},rulecolor=\color{black},
  basicstyle=\footnotesize\ttfamily,numberstyle=\tiny\color{mygray},
  captionpos=b, % sets the caption-position to bottom
  keepspaces=true, % keeps spaces in text
  numbers=left, numbersep=5pt, showspaces=false,showstringspaces=true,
  showtabs=false, stepnumber=2, tabsize=2, title=\lstname
}
\lstdefinelanguage{JavaScript}{ % JavaScript ist als einzige Sprache noch nicht vordefiniert
  keywords={break, case, catch, continue, debugger, default, delete, do, else, finally, for, function, if, in, instanceof, new, return, switch, this, throw, try, typeof, var, void, while, with},
  morecomment=[l]{//},
  morecomment=[s]{/*}{*/},
  morestring=[b]',
  morestring=[b]",
  sensitive=true
}

% Diese beiden Pakete müssen zuletzt geladen werden
%\usepackage{hyperref} % Anklickbare Links im Dokument
\usepackage{cleveref}

% Daten für die Titelseite
\title{\textbf{\Huge\Aufgabe}}
\author{\large Team: \large \TeamId \\\\
	    \large Schule: \large \TeamName \\\\
	    %\LARGE Bearbeiter/-innen dieser Aufgabe: \\ 
	    %\LARGE \Namen\\\\}
	    }
\date{\large\today}

\usepackage[]{caption}
\captionsetup{%
  figurename=Abb.,
  tablename=tab.
}

\begin{document}

\maketitle
%\tableofcontents


In einem Parallelogramm sind die gegenüberliegenden Winkel gleich groß. Außerdem ist bekannt, dass die zwei Paare von Dreiecken, die durch die Diagonalen im Parallelogramm entstehen, jeweils kongruent sind. Dies kann durch den Kongruenzsatz WSW (Winkel-Seite-Winkel) nachgewiesen werden. In diesem Fall wären das die Dreiecke \(\Delta CBM\) und \(\Delta DAM\) bzw. \(\Delta DMC\) und \(\Delta ABM\), was bedeutet, dass auch ihre Innenwinkel gleich sind. Die Strecke EM ist eine Sehne im ersten (unteren) Kreis. Über die Sehne EM liegen zwei Peripheriewinkel, die nach dem Peripheriewinkelsatz gleich groß sind: \(\angle MBE\) und \(\angle MAE\). 
\\
\(\angle MBE\) = \(\angle MAE\) 
\\

Daraus folgt, dass die Winkeln \(\angle MBE\) und \(\angle MCB\) gleich groß sind, da, wie gezeigt, der Winkel \(\angle MAD\) gleich dem Winkel \(\angle MCB\) ist.
\\
\(\angle MBE\) = \(\angle MCB\)
\\

Der Sehnensatz über die Sehnen MB und AE besagt dasselbe. Wenn man diesen umkehrt, erkennt man, dass  \(\angle AEB\) und \(\angle AMB\) gleich groß sind. Da \(\angle AMB\) und \(\angle BMC\) supplementär sind und \(\angle AEB\) und \(\angle BED\) auch, und \(\angle AEB\) und \(\angle AMB\) gleich sind, müssen auch \(\angle BMC\) und \(\angle BED\) gleich groß sein. 
\\
\(\angle AEB\) = \(\angle AMB\)\\
\(\angle BMC\) = \(\angle BED\)
\\

Die Winkeln \(\angle EMB\) und \(\angle BAE\) sind laut dem Sehnensatz supplementär. Dies heißt, dass der Winkel \(\angle DME\) gleich dem Winkel \(\angle BAE\) ist, da \(\angle DAM\) und \(\angle EMB\) auch supplementär sind. Durch den Peripheriewinkelsatz über die Sehne ED ergibt sich, dass die Winkel \(\angle DME\) und \(\angle DFE\) gleich groß sind. Da nun gezeigt wurde, dass \(\angle DCB\) gleich \(\angle BAE\) gleich \(\angle DME\) gleich \(\angle DFE\) ist, folgt daraus, dass \(\angle DCB\) gleich groß wie \(\angle DFE\) ist.
\\
\(\angle DCB\) = \(\angle DFE\)
\\

Daraus folgt, dass \(\angle DCB\) und \(\angle BFD\) supplementär sind. Wenn man jetzt das Viereck EBCD betrachtet und weiß, dass \(\angle DCB\) und \(\angle BFD\) supplementär sind, heißt es, dass auch \(\angle FDC\) und \(\angle CBE\) supplementär sind und dieses Viereck deshalb ein Sehnenviereck ist.
\\
EBCD -> Sehnenviereck\\

Deshalb kann man jetzt in diesem Sehnenviereck den Peripheriewinkelsatz erneut anwenden, und zwar an der Sehne FD, woraus sich ergibt, dass \(\angle DBF\) gleich groß wie \(\angle DCF\). Da ich schon gezeigt habe, dass die Winkel \(\angle DBF\) und \(\angle ACB\) gleich groß sind, folgt daraus, dass \(\angle ACB\) gleich \(\angle DCF\), w.z.b.w.
\end{document}




